%\input{tcilatex}
%\input{tcilatex}
%\input{tcilatex}


\documentclass[12pt]{article}
%%%%%%%%%%%%%%%%%%%%%%%%%%%%%%%%%%%%%%%%%%%%%%%%%%%%%%%%%%%%%%%%%%%%%%%%%%%%%%%%%%%%%%%%%%%%%%%%%%%%%%%%%%%%%%%%%%%%%%%%%%%%%%%%%%%%%%%%%%%%%%%%%%%%%%%%%%%%%%%%%%%%%%%%%%%%%%%%%%%%%%%%%%%%%%%%%%%%%%%%%%%%%%%%%%%%%%%%%%%%%%%%%%%%%%%%%%%%%%%%%%%%%%%%%%%%
\usepackage{amsmath, amssymb, amsfonts}
\usepackage{graphicx}
\usepackage{geometry}
\usepackage{enumitem}
\usepackage{bm}
\usepackage{hyperref}

\setcounter{MaxMatrixCols}{10}
%TCIDATA{OutputFilter=Latex.dll}
%TCIDATA{Version=5.50.0.2953}
%TCIDATA{<META NAME="SaveForMode" CONTENT="1">}
%TCIDATA{BibliographyScheme=Manual}
%TCIDATA{LastRevised=Tuesday, June 10, 2025 18:50:48}
%TCIDATA{<META NAME="GraphicsSave" CONTENT="32">}

\geometry{margin=1in}

%\input{tcilatex}

\begin{document}

\title{Buera \& Shin (2013) -- Financial Friction and the Persistence of
History: A Quantitative Exploration}
\author{Alessandro Di Nola and Robert Kirkby}
\date{June 2025}
\maketitle

\section*{Introduction}

Brief description of (incomplete) replication of Buera and Shin (2013). I
denote the exogenous shock $z$, which they call $e$, and I also call $\gamma 
$ what they call $\sigma $. I use $\mu $ as the agent distribution (while
they had it as the distribution of just the exogenous process, they used $G$
for the agent distribution).

The model is about solving a transition path in an infinite horizon model.
Here I just write the steady-state problem.

\section*{Value Function Formulation}

Buera and Shin (2013) present the household problem as a sequence problem.
We rewrite it as a value function problem:

\begin{equation}
V(a,z)=\max_{c,a^{\prime }}\left\{ \frac{c^{1-\gamma }}{1-\gamma }+\beta 
\mathbb{E}[V(a^{\prime },z^{\prime })|z]\right\}  \label{eq:bellman}
\end{equation}%
subject to 
\begin{equation*}
c+a^{\prime }\leq \max \{w,\pi (a,z|r,w)\}+(1+r)a
\end{equation*}%
and%
\begin{equation*}
a^{\prime }\geq 0
\end{equation*}%
where

\begin{equation*}
\pi (a,z|r,w)=\max_{l,k}\left\{ f(z,k,l)-wl-(r+\delta )k\right\} \text{
subject to }k\leq \lambda a
\end{equation*}%
and%
\begin{equation*}
f(z,k,l)=z(k^{\alpha }l^{1-\alpha })^{1-\nu }
\end{equation*}

This reduces to a one-state dynamic programming problem since the profit
part can be solved analytically. Solving the VFI problem delivers the policy
functions $a^{\prime }(a,z)$, $e=e(a,z)$, $e\in \left\{ 0,1\right\} $, where
optimal occupation choice is 
\begin{equation*}
e(a,z)=\left\{ 
\begin{array}{cl}
1 & \text{if }\pi (a,z|r,w)>w \\ 
0 & \text{otherwise.}%
\end{array}%
\right.
\end{equation*}

\section*{Analytical Solution}

\begin{equation*}
k_1 = \left[ \left( \frac{1}{r + \delta} \alpha(1 - \nu)z \right)^{1 - (1 -
\alpha)(1 - \nu)} \left( \frac{1}{w}(1 - \alpha)(1 - \nu)z \right)^{(1 -
\alpha)(1 - \nu)} \right]^{1/\nu}
\end{equation*}
\begin{equation*}
k^* = \min\{k_1, \lambda a\}
\end{equation*}
\begin{equation*}
l^* = \left[ \frac{1}{w}(1 - \alpha)(1 - \nu)z(k^*)^{\alpha(1 - \nu)} \right]%
^{\frac{1}{1 - (1 - \alpha)(1 - \nu)}}
\end{equation*}
\begin{equation*}
\pi(a, z|r, w) = z(k^*)^\alpha (l^*)^{1 - \alpha})^{1 - \nu} - wl^* - (r +
\delta)k^*
\end{equation*}

\section*{General Equilibrium Conditions}

There are two GE prices: interest rate $r$ and wage rate $w$.

Labor market clearing: 
\begin{equation*}
\int l^{\ast }(a,z)\cdot \mathbb{I}_{\left\{ e(a,z)=1\right\} }\,d\mu
(a,z)=\int \mathbb{I}_{\left\{ e(a,z)=0\right\} }\,d\mu (a,z)
\end{equation*}

Capital market clearing: 
\begin{equation*}
\int k^{\ast }(a,z)\cdot \mathbb{I}_{\left\{ e(a,z)=1\right\} }\,d\mu
(a,z)=\int a\,d\mu (a,z)
\end{equation*}

Implied by Walras law:%
\begin{equation*}
\int c(a,z)d\mu (a,z)+\int a\,d\mu (a,z)=\int \left[ \mathbb{I}_{\left\{
e(a,z)=0\right\} }\cdot w+\mathbb{I}_{\left\{ e(a,z)=1\right\} }\cdot \pi
(a,z)\right] d\mu (a,z)+(1+r)\int a\,d\mu (a,z)
\end{equation*}%
which simplifies to%
\begin{eqnarray*}
C &=&w\int \mathbb{I}_{\left\{ e(a,z)=0\right\} }\,d\mu (a,z)+\int \mathbb{I}%
_{\left\{ e(a,z)=1\right\} }\pi (a,z)d\mu (a,z)+rK \\
&=&wL+Y-wL-\left( r+\delta \right) K+rK \\
&=&Y-\delta K
\end{eqnarray*}

\section*{Calibration}

\noindent \textbf{External parameters}. $\gamma =1.5$, $\delta =0.06$, $%
\alpha =0.33$, $\lambda =\infty $

\noindent \textbf{Internal calibration}. Set $v$ (span of control), $\eta
,\psi $ (dispersion and persistence of entrepreneurial ability) and $\beta $
(discount rate) to match:

\begin{itemize}
\item Top 10\% of employment ($v$, $\eta $)

\item Top 5\% of earnings ($v$, $\eta $)

\item Entrepreneurs exit rate ($\psi $)

\item Real interest rate ($\beta $)
\end{itemize}

\noindent \textbf{Stochastic process}. Entrepreneurial ability $z$ follows a 
\textit{truncated} Pareto distribution with density function 
\begin{equation*}
p(z)=\left\{ 
\begin{array}{c}
\eta z^{-\left( \eta +1\right) }\text{ if }z\geq 1 \\ 
0\text{ \ \ \ \ \ \ otherwise}%
\end{array}%
\right.
\end{equation*}%
Each period, an individual keeps his old ability $z$ with probability $\psi $%
. With complementary probability $1-\psi $ he draws a new ability $z^{\prime
}$ from $p(z^{\prime })$ given above. Clearly, the parameter $\psi $ governs
the persistence of the stochastic process for ability, whereas $\eta $
governs the dispersion. Given these assumption for the stochastic process of
entrepreneurial ability, the continuation value in (\ref{eq:bellman}) can be
written more explicitely as follows:%
\begin{equation*}
E\left[ V(a^{\prime },z^{\prime })|z\right] =\psi V(a^{\prime },z)+\left(
1-\psi \right) \int V(a^{\prime },z^{\prime })p(z^{\prime })dz^{\prime }.
\end{equation*}%
In order to re-use the general routines for value function iteration and
distribution iteration, it is convenient to \textit{recast the stochastic
process described above as a Markov chain}. First, we assume we have already
discretized the Pareto distribution as $z\in \left\{ z_{1},z_{2},\ldots
,z_{n}\right\} $ and $p(z)=\left[ p_{1},\ldots ,p_{n}\right] $ (a $n\times 1$
\textit{column} vector such that $\sum_{i}p_{i}=1$). Then, we observe that
if the current ability shock is $z_{i}$, then next-period ability shock will
be $z_{i}$ with prob $\psi +(1-\psi )p_{i}$ and $z_{j}$ with prob $(1-\psi
)p_{j}$, for $j\neq i$. It is easy to check that $\sum \pi (z,z^{\prime
})=\psi +(1-\psi )p_{i}+\sum_{j\neq i}(1-\psi )p_{j}=1$, i.e. each row $i$
of the Markov chain sums to one, for all $i=1,\ldots ,n$. Therefore the $%
n\times n$ Markov chain for the shock $\left( z,z^{\prime }\right) $ can be
written as%
\begin{equation*}
\Pi _{n\times n}=\psi \mathbf{I}_{n}+(1-\psi )\mathbf{1}_{n}p^{T}
\end{equation*}%
or, more in detail, as%
\begin{equation*}
\Pi _{n\times n}=\psi 
\begin{bmatrix}
1 & 0 & \ldots & 0 \\ 
0 & 1 & \ldots & 0 \\ 
\vdots & \vdots & \ddots & \vdots \\ 
0 & 0 & 0 & 1%
\end{bmatrix}%
+(1-\psi )%
\begin{bmatrix}
1 \\ 
1 \\ 
\vdots \\ 
1%
\end{bmatrix}%
\begin{bmatrix}
p_{1} & p_{2} & \ldots & p_{n}%
\end{bmatrix}%
\end{equation*}%
where $\mathbf{I}_{n}$ is the $n\times n$ identity matrix and $\mathbf{1}%
_{n} $ is an $n\times 1$ \textit{column} vector of 1s.

Note: For the replication I used the files \texttt{support.dat} and \texttt{%
dist.dat} kindly provided by the authors. If I replicate the discretization
as described in their paper, I get values for $z $ and $p(z) $ which are
quite different from the ones used in the paper.

\section*{Appendix: Profit Maximization Derivations}

\begin{equation*}
\pi(a, z|r, w) = \max_{l, k} \left\{ z(k^\alpha l^{1 - \alpha})^{1 - \nu} -
wl - (r + \delta)k \right\}, \quad \text{s.t. } k \leq \lambda a
\end{equation*}

\paragraph{First Order Conditions (Unconstrained):}

\begin{equation*}
\frac{\partial \pi}{\partial l} = (1 - \alpha)(1 - \nu) z k^{\alpha(1 -
\nu)} l^{(1 - \alpha)(1 - \nu) - 1} - w = 0
\end{equation*}
\begin{equation*}
\frac{\partial \pi}{\partial k} = \alpha(1 - \nu) z k^{\alpha(1 - \nu) - 1}
l^{(1 - \alpha)(1 - \nu)} - (r + \delta) = 0
\end{equation*}

Solving gives: 
\begin{equation*}
l^* = \left[ \frac{1}{w} (1 - \alpha)(1 - \nu) z k^{\alpha(1 - \nu)} \right]%
^{\frac{1}{1 - (1 - \alpha)(1 - \nu)}}
\end{equation*}
\begin{equation*}
k_1 = \left[ \left( \frac{1}{r + \delta} \alpha(1 - \nu)z \right)^{1 - (1 -
\alpha)(1 - \nu)} \left( \frac{1}{w}(1 - \alpha)(1 - \nu)z \right)^{(1 -
\alpha)(1 - \nu)} \right]^{1/\nu}
\end{equation*}
\begin{equation*}
k^* = \min\{k_1, \lambda a\}
\end{equation*}

\end{document}
